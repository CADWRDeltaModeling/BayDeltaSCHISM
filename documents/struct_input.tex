


\section{Model Input for Structures}
\subsection{Enabling Structures ({\em param.in})}
Hydraulic stuctures are enabled in \url{param.in} using the parameter {\em ihydraulics}. This is a binary
flag with 1 indicating that structures are to be used and 0 indicating that they will be ignored.

\subsection{Defining Structures ({\em hydraulics.in})}
Below is an annotated example \url{hydraulics.in} file. The line numbers are not part of the original file
and the comments (parts after the "`!"') are not required.

\begin{samepage}
\verbatiminput{hydraulics.in}
\end{samepage}
\subsubsection{Global header (example line 1-2)}
The first two lines of \url{hydraulics.in} include two global parameters, 
the total number of structures and the nudging factor.

The gate equations are not enforced exactly. We have found that this specification is not fully stable, particularly when gates
are abruptly installed or open. Instead we use a relaxation formulation:

$$Q(t+\frac{\Delta t}{2})= (1-\chi) Q(t) + (\chi) Q_{\text{s}}(\eta(t),\mathbf{u}(t),\phi(t+1))$$

where $Q_s$ is the explicit flow calculation for the structure based on state variables and parameters at time $t$.

In other words, rather than being set exactly to the gate value, the (coupled) flow boundaries will be nudged a fraction $\chi$ towards this value. 
The nudging factor can also be interpreted as a time constant. Using a small value like 0.1 will provide maximum stability, but is too slow
to respond to tidal fluctuations. 

\subsubsection{Structure geometry}
After the global parameters, the next lines (3-10 in the example) represent the identification and geometry of the structure:
\begin{enumerate}
\item [Line 3] An index and name for the structure. The indices must be sequential and the maximum length of name is 32 characters
\item [Line 4] The number of node pairs in the definition, the upstream reference node and the downstream reference node. All use global node numbers.
\item [Line 5-10] For each node pair in the string defining the structure, the member of the pair on the upstream and downstream side. 
The start and end must be on land boundaries. The concept of a node pair is illustrated by the unprimed and primed nodes in Figure \ref{fig:structmesh}
\end{enumerate}

\subsubsection{Parameters and coefficients}
\label{sec:gate_spec}
After the geometric information are some parameters that are specific to the structure type.
In the example, this happens to fall on lines 11-14 although this would be different with different geometry
or in a \url{hydraulics.in} with multiple gates.
The first of these lines controls this input with the structure type being listed on the first of these lines. 
All parameters are specified {\em per unit}. For all structures except the hydraulic tranfer, 
the number of duplicate units is controlled by the variable {\em nduplicate}. 

Many of the structures have invert elevations among their parameters. The datum for this elevation is the same
as the datum for the elevation state variable $\eta$ in the model.

\paragraph{transfer}
A transfer is a coupled boundary condition (outflow and inflow). The only parameter is the prescribed flow.
\begin{verbatim}
struct_type        ! type of structure (transfer)
nduplicate         ! number of duplicate units
flow               ! flow in cms
\end{verbatim}

\paragraph{orifice,radial}
\begin{verbatim}
struct_type        ! type of structure (orifice, radial)
nduplicate         ! number of duplicate units
elev width height  ! invert elevation (m), width, height (rectangular)
coef op_down op_up ! flow coeficient, operating coefficient down and up
\end{verbatim}

\paragraph{radial\_rh}
A {\em radial\_rh} gate has a flow coefficient that has both
a constant term and a linear variation in the gate height:
\begin{verbatim}
struct_type        ! type of structure (radial_rh)
nduplicate         ! number of duplicate units
elev width height  ! invert elevation (m), width (m), height (m)
coef coef_linear   ! constant, linear height-based parameters of flow coef.
op_down op_up      ! op coef down/up
\end{verbatim}

\paragraph{weir}
An orifice is submerged, rectangular orifice (outflow and inflow). 
\begin{verbatim}
struct_type        ! type of structure (weir)
nduplicate         ! number of duplicate units
elev width height  ! invert elevation, width of  a single unit, height
coef op_down op_up ! flow coef., operating coefficients down and up
\end{verbatim}

\paragraph{culvert}
An culvert is a round culvert
\begin{verbatim}
struct_type        ! type of structure (culvert)
nduplicate         ! number of duplicate units
elev width         ! invert elevation, radius
coef op_down op_up ! flow coeficient, operating coeficient down and up
\end{verbatim}

\paragraph{weir\_culvert}
This is a combination of weir and culvert
\begin{verbatim}
struct_type           ! type of structure (culvert)
weir_nduplicate       ! number of duplicate weir units
weir_elev weir_width  ! invert elevation, radius
weir_coef weir_op_down weir_op_up ! flow, operating coefs for weir
pipe_nduplicate       ! number of duplicate culvert/pipe units
pipe_elev pipe_width              ! invert elevation, radius
pipe_coef pipe_op_down pipe_op_up ! flow, operating coefs for culvert
\end{verbatim}

\subsubsection{Enabling time series control}
\label{sec:timeflag}
The final line (line 15 in the example) of each structure definition is a flag (1=True, 0=False) indicating whether a time history (\url{*.th}) file will be used to make many of the parameters time-varying -- effectively replacing the values loaded in 
\url{hydraulics.in}. Time series control is covered in Section \ref{sec:timeseries}.

\subsection{Time Series Files ({\em *.th})}
\label{sec:timeseries}
If you set the time series flag for a gate coefficient to 1=True as indicated in \ref{sec:timeflag}, time series control is
enabled for the gate. You need to provide an input file named \url{[gate_name].th} where {\em [gate\_name]} 
is identical to the name used in the gate definition.

The file \url{[gate_name].th} is a multivariate time history.As with all \url{*.th} files the first column  is time in elapsed seconds since the start of the run and the other columns are space delimited.
One quirk compared to other SCHISM inputs is that the times can be irregular and interpolation 
for gate variables is based on constant repetition of the previous value, not on linear interpolation. This, we feel, is more typical of the actuator on a gate; some things (like "`installation"') are
also not meaningful at intermediate values.

The other parameters that are included depend on the gate type, and the columns for each type are given below. 
Most of the variables are identical to the ones listed in \url{hydraulics.in} and described in
Section \ref{sec:gate_spec}. Howerver, there is one special variable {\em install} that is always 
located in the column immediately after the time column. This variable takes on integer values and must be zero or one. 
Setting the installation to zero removes the structure, restoring the original algorithm.
 
\begin{description}
\item[transfer] time  install(int) flow    ! Install = \(0,1\)
\item[culvert,weir] time install(int) nduplicate(int) op\_down op\_up elev width
\item[orifice,radial,radial\_rh] time install(int) nduplicate(int) op\_down op\_up elev width height
\item[weir\_culvert] time install (int) ndup\_weir (int) op\_down\_weir op\_up\_weir elev\_weir width\_weir ndup\_pipe (int) down\_op\_pipe up\_op\_pipe elev\_pipe radius\_pipe
\end{description}

Below is a sample \url{dcc.th}, corresponding to the radial gate used in the original example \url{hydraulics.in}:
\begin{samepage}
\verbatiminput{dcc.th}
\end{samepage}


