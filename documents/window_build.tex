
\documentclass[12pt]{report}
\usepackage{graphicx}
\usepackage{listings}
\usepackage{float}
\usepackage[titletoc]{appendix}
\usepackage{hyperref}
\usepackage{url}
\usepackage{bbding} 
\usepackage{textcomp} 

\tolerance=600

\floatstyle{plain} % optionally change the style of the new float
\newfloat{Code}{H}{myc}
\lstloadlanguages{Python}

\begin{document}
\lstset{basicstyle=\ttfamily\tiny}


This document is mainly intended as a guide to build selfe in windows sysem

To build selfe there are several requirements:
\begin{enumerate}
\item MS-HPC 2008 must be installed.
\item Netcdf C and fortran libs and include folder are available in the system. 
      We have build  Netcdf4.3.2 C and fortran libs and include folder for distribute
			also.
\item METIS and PARMETIS libs and include folder are available in the system.
      We have build  METIS and PARMETIS libs and include folder for distribute
			also.
\item MS-Visual Studio 2010.
\item CMAKE 2.8 and above for Windows.
\item Selfe source.
\end{enumerate}


Steps of build is follows:

\begin{enumerate}
\item  Starts CMAKE GUI, and Input the path of selfe source code on the first textbox on the interface and the location where visual studio building project to be placed.

\item Click the {\bf Configure} button on the very bottom of the interface. There will be a popup window showing up to allow user choose 
the type of building  project. Select \emph{visual studio 2010 x64}.

\item Then there will be a number error message from CMAKE, for it can't
find some required lib and including path. User need to configure them 
manually.  Here are the entries user need to configure.

\begin{table}
	\centering
		\begin{tabular}{|c|c|r|}
\hline
Entry Name & Example \\
\hline \hline
NetCDF\_C\_LIBRARY       & D:/devtools/netCDF432/lib/netcdf.lib \\
NetCDF\_Fortran\_LIBRARY & D:/devtools/netCDF432/lib/fortran/ncfotran.lib \\
NetCDF\_INCLUDE\_DIR     & D:/devtools/netCDF432/include \\
PARMETIS\_DIR            & D:/devtools/parmetis-4.0.3\\
PARMETIS\_INCLUDE\_DIR   & D:/devtools/parmetis-4.0.3/include\\
PARMETIS\_LIBRARY        & D:/devtools/parmetis-4.0.3/lib/parmetis.lib\\
METIS\_LIBRARY           & D:/devtools/parmetis-4.0.3/libmetis/Release/metis.lib\\
\hline
		\end{tabular}
	\caption{CMake entries be configured manually}
	\label{tab:selfe}
\end{table}

In addition, user need to check all the {\bf CMAKE\_Fotran\_FLAGS} and make sure the byte record assumption is  \emph{-assume:byterecl}  instead of
\emph{-assume byterecl}, for the former one is correct syntax in the Windows environment. 


\item Click the button {\bf Generate} to create Visual Studio Solution.

\item Open the resulting Visual Studio Solution. User need to increase the reserved stack size for the pelfe to avoid the possible error of stack overflow 
when running it. Right click project {\bf pelfe}, Go to {\bf Properties}, Select {\bf Linker}, and then {\bf System}. On the table to the right, fill in the 
number (in kilo byte) for item {\bf Stack Reserve Size} and {\bf Stack Commit Size}. The two number can be the same and less than the system memory.


\end{enumerate}


We have successfully building 64bit netCDF-fortran library in Windows based on netCDF4.3.2-C library. The netCDF4.3.2-C library can be downloaded from   \url{http://www.unidata.ucar.edu/software/netcdf/docs/winbin.html}.  The source code of netCDF-fortran can be found on
\url{https://github.com/Unidata/netcdf-fortran/releases/}. This source code provides a CMakeLists.txt to help user creating Visual Studio
project in Windows.  Here are the simple steps of building netCDF-fotran library:

\begin{enumerate}
\item  Starts CMAKE GUI, and Input the path of netCDF-fortran source code on the first textbox on the interface and the location where visual studio building project to be placed.

\item Click the {\bf Configure} button on the very bottom of the interface. There will be a popup window showing up to allow user choose 
the type of building  project. Select \emph{visual studio 2010 x64}. 

\item CMake may not find out the location your netCDF-C library. In such a case, user need
to set up them manually. The table below show a example,

\begin{table}
	\centering
		\begin{tabular}{|c|c|r|}
\hline
Entry Name & Example \\
\hline \hline
NetCDF\_C\_LIBRARY          & D:/devtools/netCDF432/lib/netcdf.lib \\
NetCDF\_C\_INCLUDE\_DIR     & D:/devtools/netCDF432/include \\
\hline
		\end{tabular}
	\caption{NetCDF entries be configured manually}
	\label{tab:netCDF}
\end{table}

\item Click the button {\bf Generate} to create Visual Studio Solution.

\item Open the resulting Visual Studio Solution. There should be a project named \emph{ncfortran}. Right click it, ao to {\bf Properties}, Select {\bf C/C++}, and then {\bf Preprocessor}. On the table to the right, Click entry \emph{Preprocessor Definitions}, User need to make sure those Marco are defined as the order below:
\begin{enumerate}
\item netcdff\_EXPORTS
\item DLL\_NETCDF
\item USE\_NETCDF4
\item NC\_DLL\_EXPORT
\end{enumerate}
   

\item Build project ncfotran and copy the resulting netCDF fortran libary to some location.

\end{enumerate}


The ParMetis in the selfe source doesn't compiled in Windows. User can download the latest ParMetis source code from \url{http://glaros.dtc.umn.edu/gkhome/metis/parmetis/download}. Building ParMetis and Metis library in Windows using CMAKE and Visual Studio
is quite straightforward. 

\end{document}