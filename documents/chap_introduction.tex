\chapter{Introduction}
This document describes the first calibration of the three dimensional \gls{selfe}
on the Bay-Delta, performed collaboratively by the California Department of Water Resources 
and the Virginia Institute of Marine Sciences. The main items of discussion will be the scope of the Bay-Delta SELFE effort, 
the assumptions and additions made for the domain, initial calibration and validation results, 
and a statement of suitability for various types of current and future studies. 

Our goal for the initial calibration was to accurately illustrate and answer operational and planning 
questions involving multidimensional flow and constituent transport in the Bay-Delta system
over periods of months or years. This goal requires that we accurately resolve the principle mechanisms of multi-dimensional transport up the estuary and in Delta channels: 
gravitational circulation, tidal trapping, pumping and flood-ebb asymmetry. In many cases it also requires that the model
be reconfigurable in a {\em near field - far field} arrangement whereby a higher resolution regions of focus is nested 
within the coarser, pre-calibrated grid connecting the study region to the rest of the system.

At the time of writing the following are the motivating applications for our work:

\begin{itemize}
	\item \gls{bdcp} mandated habitat restoration and island flooding
  \item Local changes in velocity patterns caused by tunnel intakes or receiving sections of the Clifton Court Forebay.
	\item The effects of sea level rise, climate change and storm surge in the Delta
  \item Yolo Bypass fish passage
  \item Mercury fate in Liberty Island, Yolo Bypass and the Delta
  \item Full life cycle modeling of salmon (collaboration with NMFS and NASA)
\end{itemize}

In terms of resolution and detail, many of these applications share two common requirements. First the model 
must capture variation of the mean flow field over the cross section (net of turbulence). 
Second, many of these applications require local regions of local refinement or additional physics. 

SELFE is a three-dimensional, cross-scale hydrodynamic and transport model capable of modeling flow, water quality, 
temperature and particle trajectories over scales ranging from oceanic to fairly small channels. 
Its meshing is of the "unstructured"' type (see section***), allowing it to flexibly conform 
to natural bathymetry and making it amenable to near-field regions of focus. The SELFE 
community has coupled or built modules for sediment transport, wind-wave 
interaction, oil spills, biogeochemistry and particle transport (Figure ***). While features from a generic modeling suite usually
need to be adjusted to fit the sources, sinks and interactions of the San Francisco Bay-Delta systems, 
these modules represent a significant head start tackling multidisciplinary problems.  
Two of our ongoing projects are of this multidisciplinary nature: 
the salmon life cycle model collaboration with NMFS and NASA includes flow, temperature, food production and fish travel; 
The Methylmercury project involves specific chemistry as well as sediment/detritus interaction. 

The current calibration document, however, focuses on tide propagation and salinity transport. The reasons for focusing on salt are well-known. 
First, salinity intrusion is still by far the most important water quality issue facing the water projects and salt (or rather conductivity) 
is extensively monitored at stations throughout the estuary. Second, the processes giving rise to mixing are very similar 
for any diffuse, conservative tracer. The ability to model the circulation of salt is likely to predict fidelity on other constituents such as  ***. 

The remainder of this document explores the range of applicability further by the Bay-Delta calibration. The purpose of calibrating a 
hydrodynamic model is to ready the model for a new domain or application, adjust parameters and gain insight
into the model suitability for various operational and planning purposes. Calibration is always indicated for a new model application; 
it may also be needed when new physical processes (sediment, biogeochemistry) are added or when new data becomes available. 
This insight can then be better quantified by validating over a longer period in which the model is not adjusted or fit to the data. 

Chapter  *** describes the SELFE model formulation and features, including two capabilities developed for this project: 
subgrid hydraulic structures to model Delta gates and barriers and mass sources and sinks to model island diversions and returns. 
Chapter *** describes the model configuration and assumptions and configuration of the general Bay-Delta application. Chapters *** and *** respectively detail model calibration and validation results. Chapter ** summarizes these results into statements of suitability we think we have
established for the model, and gives a description of ongoing work in certain more specific (and more multidimensional) metrics of suitability that will pertain to subjects such as secondary circulation and mercury cycling on partial island flooding.

We hope this Version 1 calibration will establish a suitable benchmark of progress to date and the suitability of the model for numerous applications.



 
