\chapter{Validation}
The purpose of model validation is to assess, model accuracy. This is typically
done in a different setting than the calibration calibrated. The plots and metrics are largely identical to those introduced in 
\label{chap:validate}. The main differences are the period and selection of station data.

\section{Period of Validation}
The validation is a continuation of the 2009 simulation through *** 2011. Essentially, this is just a
continuation of the calibration simulation. No new field initial conditions are required -- we used the
stopping state of the calibration and allowed a suitable gap of *** to decouple the two period. 
Personnel carrying out the validation were the same as those doing the calibration, but the  
validation component was assembled and analyzed after the calibration was complete.

The validation period brings the model into contact with a slightly wider range of hydrology and
different gate ops ****. The validation also introduces some stations that were previously underutilized. 
Most notably, starting in late 2010 we were able to incorporate midstream USGS water quality 
instruments that were installed at flow stations starting in 2010. We also made more extensive use of USBR 
stations in the validation not because these data are better*** but because
we only acquired certified data for these sites late in the project (we had access through CDEC previous to that
but  as well as USBR stations that, while long-running, 


\section{Stage results}
\section{Flow results}
	\subsection{Flow station data}
	\subsection{Tidal phase and amplitude}
	\subsection{Net flow maps}	
	\subsection{HF Radar}
\section{Salinity results}
  \subsection{Salinity station data}
  \subsection{Surface-bottom stratification}
	\subsection{USGS Cruise CTD casts}

