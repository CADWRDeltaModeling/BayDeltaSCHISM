

\chapter{Bay-Delta Calibration for Free Surface, Flow and Salinity}

\section{Calibration Period}
The Calibration 

\section{Skill metrics for scalar station data}
For station observations of scalar quantities such as stage and flow, 
we evaluated model performance based on both visual assessments of time series plots and 
quantitative fitness scores. 

Each station analysis includes **14-day dynamic plots revealing tidal characteristics 
at the site over spring and neap tide longer ***day plots of tidally filtered results showing the
system response to longer period fluctuations such as barometric events and seasonal
inflow changes. For the tidally filtered plots only, infrequent small gaps (shorter than *** hours)
in the observed records were filled using interpolation or local-in-time averages 
in order to avoid the widening of the gaps to days that occurs when missing data are filtered. 

Below the time series plots are scatterplots of the phase-adjusted model results 
and observed data fitted with a regression line. The slope of the regression line has been
described by *** as a measure of signal amplitude amplification and in a more 
general sense a measure of conditional bias.  The correlation coefficient 
is reported as well but not prominently. For serially correlated data 
this statistic occupies a narrow scale \cite{Ralston2006}, neglects important aspects of fit
\citep{Murphy88}, may give false positives or high scores for nearly random predictions \citep{Ralston2006}
and does not appear to be informative in the presence of patterned errors. 

The scale of the scatter plots gives a good indicator of the total variability in the data 
including both tidal and non-tidal components. It can be contrasted with the scale of the 
tidally-filtered time series plots, whose y-axis is a good indicator of the variability of
the subtidal part of the signal. The scale of the dynamic plots is chosen such that ****. 
This is a rather important choice and it results in a somewhat "`zoomed in"' view of the tidal 
signal compared to ***, while

We also evaluated fitness scores for important aspects (bias, fit, skill, phase) of 
model performance. The following statistics are reported where
appropriate:

\begin{description}
  \item{{\textbf{RMSE}}} Root mean square error. Note that unlike the other statistics 
	below marked with a $\phi$ subscript (and published calibrations from other models), 
	the data are not phase-corrected (***).
  \item{{\textbf{Lag}}} An estimation of lag based on cross-correlation analysis, 
	similar to that described in ****RMA. The idea of this analysis is to shift
	the model results forward and backward in time until the maximum cross-correlation 
	between the two series is obtained. A positive lag indicates that the model trails the
	field observations in the timing of tidal signals. The effectiveness
	of this statistic depends on the strength of the tidal signal compared to noise and longer term trends with vague     
	periodicity. The analysis was not performed for Delta salinity stations.
	\item{\textbf{Bias}\subphi} The median bias  of phase corrected error (modeled - observed)
	\item{\textbf{NSE}\subphi} The Nash-Sutcliffe efficiency for the phase-corrected error:
	\begin{gather}
	NSE\subphi(f,r,x) = 1 - (MSE(f,x)/MSE(r,x)) \\
	\text{where} ~ MSE(f,x) = \frac{1}{n} \sum\limits_{i=1}^{n}(f_i - x_i)^2 \\
	\end{gather} is the mean squared error between a forecast ($f$), an observation ($x$) 
	and a {\em reference forecast}. The choice of 
	reference forecast can several possible skill scores -- 
	the Nash-Sutcliffe is the simplest with $r$ representing the station time-average.
	\item{\textbf{r}\subphi} The correlation coefficient (the "`r"' in $R^2$ for the fit 
	of the observations on the phase-corrected model values.
\end{description}


In addition to covering the major facets of accuracy in a tidal
system, the main criteria for these metrics were robustness and consistency with
other Bay-Delta model calibrations. The variety of published skill metrics for hydraulic and hydrologic models is extensive 
(some examples are \citet{Willmott82,Willmott85,Willmott12}, \cite{Stow09}, \cite{Murphy88}, \cite{Ralston2006}). 
No standard has emerged \citep{Stowe09}, and there seems to be some schism between practicioners who work with 
measures of accuracy (e.g. RMSE) and those who work with skill. 

For systems with periodic forcing such as an estuary, many of the above statistics are sensitive to the phase 
(time shift) of the model relative to the observations. If this error is not controlled, 
it reduces the diagnostic value of the suite because the statisitics would not differentiate an excellent
tidal signal with a modest shift from a shabby tidal signal. 
Following  RMA***, we first estimate phase error and then correct for phase (by shifting the model output in time) 
when generating the remaining statistics except for RMSE which is left as a composite indicator of overall error.

The metrics are not decomposed into tidal and subtidal time scales. This means that the diagnosticthat the underlying physics and time series plots do. Subtidal and tidal information can therefore
obscure one another. In particular, tidal signals are generally much larger than the variation
of longer period fluctuations. When the two are mi

\section{Station inventory and descriptions}
The locations of stage, flow and salinity stations is shown in Figures ***. We relied on data from the following agencies and programs:

\begin{itemize}
	\item NOAA (automated retrieval)
	    \subitem Verified Water Levels from ***
			\subitem Conductivity at *** stations in the Bay.
	\item USGS (request)
	    \subitem Conductivity from the Water Quality and Sediment *** Group
			\subitem Flow, stage and (post-2010) water quality from the Delta Hydrodynamics Group in West Sacramento
			\subitem CTD (conductivity-temperature-depth) cast data from cruises conducted by ***
  \item CenCOOS surface HF Radar
	\item DWR North Central Regional Office or NCRO (Water Data Library) 
	    \subitem Flow
			\subitem Surface Water Monitoring Program
			\subitem Water Quality Monitoring Program
	\item DWR Compliance* and IEP (request) for conductivity monitoring at **** sites
	\item DWR O\&M (CDEC) for operational data and pumping rates
	\item USBR (CDEC) conductivity and temperature stations in the Delta
\end{itemize}

In each case we report the station name, agency identification and CDEC identification for familiarity. 


\section{Calibration}
\subsection{Stage}
  \subsection{Stage monitoring stations}
	\subsection{Tidal phase and amplitude}
\subsection{Flow}
  \subsection{Flow monitoring stations}
	\subsection{Net flow}
	\subsection{Tidal phase and amplitude}
\subsection{Salinity}
  \subsection{Surface salinity stations}
  \subsection{Surface-bottom stratification}
	\subsection{USGS Cruise CTD casts}
\subsection{Salinity CTD casts}

\section{Model sensitivity}
  \subsection{Mesh change}
	\subsection{Mesh refinement}
	\subsection{Friction change}
