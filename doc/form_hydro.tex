\subsection{Equations of motion}\label{sec-1}
The formulation of SELFE is based on classic expressions of mass and momentum conservation within a shallow fluid,
as well as the transport equations for salt and heat. The three dimensional expressions of these laws of motion
involve much less reduction and averaging than one or two-dimensional models. However, 3D estuary-scale models
do invoke a standard set of assumptions that will be discussed in section \ref{sec-assump}.

The SELFE variant used in this equation is defined in a Cartesian (ie map projection) frame with
bathymetric depth $h$ and free surface $\eta$ defined relative to a fixed datum (NAVD 88 in our case) as shown 
in Figure \ref{fig:depths}. The total depth $H$ is the sum of bathymetric depth and free surface elevation.



The 3D continuity, depth-integrated mass conservation, and horizontal momentum conservation equations are given by:
\begin{align}
  %& \text{3D Continuity}& \nonumber \\
  \nabla \cdot \bs{u} &= 0 \label{3Dcon}&\\
  %& \text{Depth-integrated Continuity}& \nonumber \\
  \eta_t+\nabla \cdot \int_{-h}^\eta \bs{u} & = 0 \label{cont1} &\\
  %& \text{Momentum}& \nonumber \\
  \frac{D\bs{u}}{Dt} &= \explain{ -f\bs{k} \times \bs{u} }{\text{Coriolis}} 
	                      \; \explain{-\frac{1}{\rho_0} \nabla p_A}{\text{Atmos. pressure}}
												\; \explain{- \frac{g}{\rho_0} \int_z^\eta \nabla \rho d \xi}{\text{Baroclinic}}
	                      \; \; \; \explain{-g\nabla \eta}{\text{Gravity wave}} & \nonumber \\
	                      \; \; \; &\explain{+ \nabla \cdot (\mu \nabla \bs{u})}{\text{Horizontal diffusion}}
												\; \; \; \; \explain{+\frac{\pd}{\pd z}(\nu \frac{\pd \bs{u}}{\pd z})}{\text{Vertical diffusion}} \label{mom1}
	\end{align}
	with wind and bed stress boundary conditions at the surface and bottom of the water column:
	\begin{align}
	\nu \frac{\pd \bs{u}}{\pd z} &= \bs{\tau}_w \mbox{ at } z=\eta &\\
  \nu \frac{\pd \bs{u}}{\pd z} &= \chi \bs{u}_b \mbox{ at } z=-h &
\end{align}
where the variables and parameters are:
\begin{align*}
&\eta(x,y,t)  &\text{free surface elevation (m)} &\\
&\bs{u}(x,y,z,t)  &\text{horizontal velocity(m/s)}  &\\
&h(x,y)  &\text{bathymetric depth(m)} &\\
&w  & \text{vertical velocity (m/s)}  &\\
&f     & \text{Coriolis factor (s\textsuperscript{-1} )} & \\
&g     & \text{gravity (m/s\textsuperscript{2})} & \\
&\rho  & \text{water density (kg/m\textsuperscript{3})} & \\
&p_A(x,y, t) & \text{atmospheric pressure at the free surface (N m\textsuperscript{2})} & \\
&\nu    & \text{vertical eddy viscosity (m\textsuperscript{2}/s)} & \\
&\mu    & \text{horizontal eddy viscosity (m\textsuperscript{2}/s) }\\
&\kappa & \text{vertical eddy diffusivity, for salt and heat (m\textsuperscript{2}/s)} & \\
\end{align*}

and the labeled processes in the momentum equations include:
\begin{description}
    \item{Coriolis} an apparent force that results from writing equations on a rotating system (the earth) in a projected (x,y) coordinate system. The Coriolis force is most important in the ocean and near coast, less so in estuary or riverine systems. 
	  \item{Atmospheric pressure} This is the horizontal variation of pressure above the water.
		\item{Gravity wave} Pressure differences due to slope in the water surface are the main force that causes tide and flood propagation. 
	  \item{Baroclinic forcing} This is the driving mechanism for density-driven (exchange) flow, as
commonly found in estuaries, of which salinity intrusion is an example. Another example is gravity (dense) underflow. Due to this force, the horizontal gradient of the density field will initiate 3D flow that moves denser water under the lighter water and will lead to a two-layer flow structure commonly found in stratified estuaries. 
\end{description}

\subsubsection{Assumptions}\label{sec-assump}
The Reynolds averaged shallow water equations include some simplifications of the raw (Navier-Stokes) equations in order
to eliminate small scale terms and make the equations more tractable.
\begin{itemize}
\item {\em Free surface} The free surface is incorporated into the equations and results from the 
divergence of the horizontal flows. This simplification avoids a complex moving boundary problem at the water-atmosphere interface.
\item {\em Reynolds averaging} SELFE solves for velocity in a time-mean sense, averaged over small scale turbulence. Mixing induced by turbulent fluctuations 
is accounted for by relating the fluctuations to mean flow properties using a set of relations called the {\em turbulence closure}. 
\item {\em Boussinesq approximation} The Boussinesq approximation simplifies terms in the governing equations by considering the the effect of density differences due to soluble tracers (salt, temperature and sediment) only
in a single buoyancy term.
\item {\em Hydrostaticity} Most models in SELFE's class have both a hydrostatic and non-hydrostatic option. In hydrostatic mode, the model only considers pressure forces on a parcel of water arising only from the weight of the water and atmospheric pressure above, neglecting vertical momentum. 
The assumption can be relaxed by enabling non-hydrostatic pressure; however, due to the required resolution and computation time, 
this feature is not used at system scales in any model in the Bay-Delta.
\item In hydrostatic mode, volume conservation is first enforced over the water column using horizontal velocities into and out of the water column. Vertical velocity is implied in the equations and later inferred in a separate step that invokes the 3D continuity equation. 
\end{itemize}

\subsubsection{Roughness and friction}
In SELFE, resistance is introduced into the water column through the combination of the bottom stress (drag)
boundary condition and vertical turbulent momentum diffusion. The mechanism is less direct than the direct 
body force used in 1-D and 2-D models. [how does SELFE2D fit here?]

There are two options in the stipulation of the drag coefficient for each spatial location:
\begin{itemize}
	\item $C_d$ may be given directly
	\item $C_d$ may be inferred using an analytical description (logarithmic decay) of velocity in a viscous boundary layer. The parametrized form leads to a formula for $C_d$: 
	$C_d=[\frac{1}{\kappa_0}log(\frac{\delta}{z_0})]^{-2}$, where $\kappa_0$ is von Karman's constant, and $\delta$ is the thickness
of the bottom layer. In this case the calibration parameter is roughness ($z_0$) rather than drag.
\end{itemize}

Once drag is characterized at the bed, it is mixed vertically up the water column through a turbulent eddy diffusion, which is
labeled "vertical viscosity" in equation (\ref{mom1}) and shown visually in figure ****. The mixing of slower water into 
faster water has the effect of slowing the faster water down. Near the bed, this viscous process is the analog of
friction in a lower dimension model. Higher in the water column, the effect of eddy diffusion and location of the velocity
maximum are harder to predict.

In terms of calibration the above formulation leads to two sets of parameters that need to be estimated. The first is the
drag coefficient, which may take the form of $C_d$ or roughness $z_0$. In addition, an eddy diffusion coefficient 
has to be supplied -- this is not stipulated directly but rather emerges from the choice of turbulence closure, described in Section \ref{sec-tur}.

\subsection{Turbulence closure}\label{sec-tur}
The final component of vertical mixing is the turbulence closure, which is used to obtain an eddy viscosity coefficient
for the momentum equations and a vertical eddy diffusivity for the transport equations. 

We use the vertical component of Umlauf and Burchard's \cite{Umlauf2003}  generic length-scale model, a separate differential 
equation which is integrated "off-line" of the other equations based on values from the previous time step. 
The model is as follows:
\beqa
  \frac{D k}{D t}&=&\frac{\pd }{\pd z}\left( \nu_k^\Psi \frac{\pd k}{\pd z} \right)
  +\nu_t M^2+\nu_t^\theta N^2 -\epsilon \\
  \frac{D \Psi}{D t}&=& \frac{\pd }{\pd z}\left( \nu_\Psi \frac{\pd \Psi}{\pd z} \right)
    +\frac{\Psi}{k}(c_{\Psi 1}\nu_tM^2+c_{\Psi 3}\nu_t^\theta N^2-c_{\Psi 2}\epsilon F_{wall})
\eeqa
with natural b.c.:
\beq
   \left\{ \begin{array}{ll}
       \nu_k^\Psi \frac{\pd k}{\pd z} &=0, \mbox{ at } z=-h, \mbox{ or } \eta \\
       \nu_\Psi\frac{\pd \Psi}{\pd z} &= \kappa_0 n\nu_\Psi\frac{\Psi}{l} , \mbox{ at } z=-h \\
       \nu_\Psi\frac{\pd \Psi}{\pd z} &= -\kappa_0 n\nu_\Psi\frac{\Psi}{l} , \mbox{ at } z=\eta \\
           \end{array}
   \right.  \label{tur1}
\eeq
and essential b.c.:
\beq
   \left\{ \begin{array}{ll}
       k&=(c_\mu^0)^{-2} \nu|\frac{\pd \bs{u}}{\pd z}|, \mbox{ at } z=-h, \mbox{ or } \eta\\
       l&=\kappa_0 \D \\
       \Psi &= (c_\mu^0)^pk^m(\kappa_0 \D)^n
           \end{array}
   \right.  \label{tur2}
\eeq
where $k$ is the turbulent kinetic energy, $l$ is the mixing length, $c_{\Psi *}$ are some constants and $\Psi=(c_\mu^0)^pk^m l^n$ is a generic
length-scale variable, and $\D$ is the distance from 'walls' (i.e. surface and bottom). 

The turbulence production and dissipation terms are:
\beqa
  M^2&=&\left( \frac{\pd u}{\pd z}\right)^2+\left( \frac{\pd v}{\pd z}\right)^2 \\
  N^2 &=&-\frac{g}{\rho_0}\frac{\pd \rho}{\pd z} \\
  \epsilon &=& (c_\mu^0)^3k^{1.5} l^{-1}
\eeqa

In the code, the natural b.c. is applied first (see the FEM formulation below), and the essential b.c. is then used to overwrite the
boundary values of the unknown, as in the GOTM code. SELFE has been directly coupled to the GOTM model to take advantage of the latter.

Once the equations for turbulent kinetic energy mixing length have been updated, the 
values of eddy viscosity and eddy diffusivity used in the momentum and transport equations are obtained from the relations: 
\beqa
  \nu &=& \sqrt{2k}s_m l \\
  \kappa &=& \sqrt{2k}s_h l
\eeqa
where $s_m$ and $s_h$ are stability functions, such as those given by \citet{Kantha94}

With this step, the specification of turbulent mixing (and indirectly the mechanism of friction) is complete.



